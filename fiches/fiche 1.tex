\documentclass[12pt,a4paper]{article}
\usepackage{style}

\subtitle{Fiche \no 1 -- Squats}

\begin{document}

\maketitle

%\thispagestyle{firstpage}
%\vspace{-30pt}



\subsection*{Exercice}
	Tenez vous debout, les pieds écartés de la largeur du bassin. Descendez ensuite comme si vous vouliez vous asseoir les fesses vers l'arrière jusqu'à ce que vos genoux décrivent un angle à $90^\circ$, puis remontez.

	\begin{center}
		\begin{tabular}{c|c}
			\textbf{Débutant} & \textbf{Expert} \\
			\hline
			2 séries de 40 répétitions & 3 séries de 40 répétitions \\
		\end{tabular}
	\end{center}

\subsection*{Objectifs}
	Excellente entrée en matière pour muscler les jambes et les abdominaux, vous améliorerez la fixité des jambes et votre position en équilibre.

\subsection*{Attention !}
	Quand vous descendez, ce sont bien les fesses qui partent vers l'arrière comme lorsque vous vous asseyez sur une chaise (normalement ça vous savez faire, ça fait 3 semaines qu'on est en confinement\dots). Attention à ce que les genoux ne dépassent pas les pieds quand vous descendez (garder la verticalité du tibia). Expirez à la descente.

\vfill
\begin{flushright}
	Bon courage à tous, on attend vos exercices en vidéo ! \phantom{Clémence et Thomas}

	Clémence \& Thomas.
\end{flushright}

\end{document}
