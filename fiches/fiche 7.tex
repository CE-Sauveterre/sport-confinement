\documentclass[12pt,a4paper]{article}
\usepackage{style}

\subtitle{Fiche \no 7 -- Quatre pattes dynamique}

\begin{document}

\maketitle

%\thispagestyle{firstpage}
%\vspace{-30pt}



\subsection*{Exercice}
	Mettez-vous à quatre pattes. Ensuite, levez simultanément un bras et la jambe opposée puis reposez-les et faites de même avec le second diagonal.

	\begin{center}
		\begin{tabular}{c|c}
			\textbf{Débutant} & \textbf{Expert} \\
			\hline
			$30$ répétitions pour chaque diagonal & $40$ répétitions par diagonal \\
		\end{tabular}
	\end{center}

\subsection*{Objectifs}
	Travaille les fessiers, le dos et les abdos pour améliorer votre tonicité et votre position à cheval.

\subsection*{Attention !}
	Gardez le bassin bien droit : celui-ci ne doit pas se tournez quand vous changez de diagonal. Idem pour le dos : il doit rester droit et ne pas se creuser.

	Prenez le temps de dérouler les mouvements pour accroître leur efficacité. Et prenez le temps de respirer !

\vfill
\begin{flushright}
	Bon courage à tous, on attend vos exercices en vidéo ! \phantom{Clémence et Thomas}

	Clémence \& Thomas.
\end{flushright}

\end{document}
